% !TeX root = cr.tex

\documentclass[12pt]{article}
\usepackage[utf8]{inputenc}
\usepackage[T1]{fontenc}
\usepackage[table]{xcolor}
\usepackage[french]{babel}
\usepackage{amsmath}
\usepackage{amssymb}
\usepackage{bookmark}
\usepackage{enumitem}
\usepackage{fancyhdr}
\usepackage{float}
\usepackage{geometry}
\usepackage{graphicx}
\usepackage{hyperref}
\usepackage{lastpage}
\usepackage{lmodern}
\usepackage{mathrsfs}
\usepackage{nccrules}
\usepackage{parskip}
\usepackage{tikz}
\usepackage{tikz-3dplot}
\PassOptionsToPackage{hyphens}{url}\usepackage{url}

\usetikzlibrary{angles,calc,decorations.pathreplacing,arrows.meta}
\tdplotsetmaincoords{70}{-60}

\title{\textbf{Projet imagerie 3D : Reconstruction d'une scène à partir d'images 2D}}

\author{
    \textsc{Duteyrat Antoine},
    \textsc{Sève Léo}
}

\date{\today}

%----------------------------------------------------------------%
%----------------------------------------------------------------%
%----------------------------------------------------------------%

\definecolor{couleur}{RGB}{0,0,0}
\pagestyle {fancy}

\makeatletter
\let\titre\@title
\let\auteurs\@author
\let\date\@date
\makeatother


%----------------------------------------------------------------%

% En-tête
\renewcommand{\headrulewidth}{1pt}
\setlength{\headheight}{45pt}
\fancyhead[L]{\titre}
\fancyhead[R]{}

% Pied de page
\renewcommand{\footrulewidth}{0.5pt}
\fancyfoot[C]{\thepage\ / \pageref{LastPage}}

%-----------------------------------------------------------------%
%-----------------------------------------------------------------%
%-----------------------------------------------------------------%

\begin{document}

%-----------------------%
%-----Page de garde-----%
\begin{titlepage}
    \begin{center}
        \vskip 1.5cm
        {\color {couleur}{\huge \bf \titre}}\\[5mm]
        \vskip 0.5cm
        \begin{figure}[h]
        \centering
        \includegraphics[width=7cm]{images/logo_tse.png}
        \end{figure}
        \vskip 1cm
        {\large {\auteurs}}
        \vskip 0.5cm
        \vfill
        \color{couleur}{\dashrule[1mm]{15cm}{0.5}}
        \vskip 0.2cm
        \date
      \end{center}
\end{titlepage}
\clearpage

\tableofcontents

\newpage

%-----------------------------------
\section{Objectif}
%-----------------------------------

L'objectif de ce projet est de reconstruire une scène 3D à partir d'images 2D prises d'une caméra arbitraire.
Pour cela, plusieurs étapes sont nécessaires :

- Calibrer la caméra (calibraton intrinsèque) par la méthode de Zhang (mire plane).

- Prendre plusieurs images d'une scène 3D avec la caméra.

- Triangulation des points 3D à partir des images 2D.

\newpage

%-----------------------------------
\section{Calibration de la caméra}
%-----------------------------------

%-----------------------------------
\subsection{Obtention de la matrice intrinsèque K}
%-----------------------------------

La calibration de la caméra est une étape cruciale pour obtenir des images 3D précises.
Elle permet de déterminer la matrice intrinsèque K, qui est utilisée pour projeter les points 3D du monde réel (repère caméra) sur l'image 2D capturée par la caméra (figure~\ref{fig:K_schema}).

\begin{figure}[H]
\centering
\includegraphics[width=10cm]{images/schema_K.png}
\caption{Illustration de la projection des coordonnées 3D du repère caméra sur l'image}
\label{fig:K_schema}
\end{figure}

Les distances entre les plans sont les suivantes :

\begin{figure}[H]
\centering
\includegraphics[width=10cm]{images/schema_K_dist.png}
\caption{Illustration des distances entre les plans de la caméra}
\label{fig:K_dist_schema}
\end{figure}

De cette illustration, on tire les égalités suivantes :
\begin{equation}
    \frac{y'}{y} = \frac{f}{z}
    \quad \text{et} \quad \frac{x'}{x} = \frac{f}{z}
\end{equation}
où :
\begin{itemize}
    \item $x'$ et $y'$ sont les coordonnées de l'image (en mm),
    \item $x$, $y$ et $z$ sont les coordonnées du repère caméra (en mm),
    \item $f$ est la distance focale de la caméra  (en mm).
\end{itemize}
\vspace{1cm}

D'où :
\begin{equation}
    y' = \frac{f \times y}{z}
    \quad \text{et} \quad x' = \frac{f \times x}{z}
\end{equation}

L'objectif est maintenant de passer des coordonnées en mm du repère $x';y'$ au repère $u;v$ de l'image (respectivement vertical et horizontal).
Pour cela, on introduit $k_u$ et $k_v$ les facteurs d'échelle (en pixels/mm), et on ajoute le décalage $u_{0}$ et $v_{0}$ (en pixels) pour obtenir les coordonnées en pixels dans l'image.
\begin{equation}
    \quad u = - k_u \times x' + u_{0} = \frac{- k_u \times f \times x}{z} + u_{0}
    \quad \text{et} \quad v = k_v \times y' + v_{0} = \frac{k_v \times f \times y}{z} + v_{0}
\end{equation}

Et en prenant $z = s$ :
\begin{equation}
    u = \frac{- k_u \times f \times x}{s} + u_{0}
    \quad \text{et} \quad v = \frac{k_v \times f \times y}{s} + v_{0}
\end{equation}
En multipliant par $s$ et en réorganisant, on obtient :
\begin{equation}
    \begin{pmatrix}
    s \times u \\
    s \times v \\
    s
    \end{pmatrix} = 
    \begin{pmatrix}
    - f \times k_u & 0 & u_{0} \\
    0 & f \times k_v & v_{0} \\
    0 & 0 & 1
    \end{pmatrix}
    \begin{pmatrix}
    x \\
    y \\
    z
    \end{pmatrix}
\end{equation}

En notant :
\begin{itemize}
    \item - $f \times k_{u} = \alpha_u$ : distance focale en pixels (en pixels),
    \item $f \times k_{v} = \alpha_v$ : distance focale en pixels (en pixels),
    \item $s = z$ : coordonnée de la caméra (en mm)
\end{itemize}

On obtient alors la matrice intrinsèque K de la caméra :
\begin{equation}
    \begin{pmatrix}
    sv \\
    su \\
    s
    \end{pmatrix} = 
    \begin{pmatrix}
    \alpha_u & 0 & u_0 \\
    0 & \alpha_v & v_0 \\
    0 & 0 & 1
    \end{pmatrix}
    \begin{pmatrix}
    x \\
    y \\
    z
    \end{pmatrix}
\end{equation}

Avec la matrice K au centre de l'opération (nous ne considérons pas le cisaillement $\gamma$, qui représente l'orthogonalité des lignes et colonnes du capteur de la caméra).
Cette matrice est utilisée pour projeter les points 3D du repère caméra sur l'image 2D capturée par la caméra.

%-----------------------------------
\subsection{Lien matrices K et A, utilisation et projection}
%-----------------------------------

L'objectif de la calibration est d'estimer les valeurs de la matrice intrinsèque K de la caméra de la forme suivante \cite{frenchidrone}\cite{wikipedia_calibration_camera} (équation~\ref{eq:K}) :

\begin{equation}
\begin{pmatrix}
\alpha_{u} & 0 & u_0 \\
0 & \alpha_{v} & v_0 \\
0 & 0 & 1
\end{pmatrix}
\label{eq:K}
\end{equation}

où :
\begin{itemize}
    \item $\alpha_{u}$ et $\alpha_{v}$ représentent la distance focale \textit{f} en pixels dans les directions verticale et horizontale respectivement,
    \item $u_0$ et $v_0$ sont les coordonnées du centre optique en pixels,
    \item La dernière ligne est utilisée pour homogénéiser les coordonnées.
\end{itemize}

Cette matrice est ensuite utilisée dans la conversion des coordonnées 3D du monde réel en coordonnées 2D de l'image, 
en modèle sténopé sans défaut d'orthogonalité (équation~\ref{eq:projection}).

\begin{equation}
\label{eq:projection}
\begin{pmatrix}
su \\
sv \\
s
\end{pmatrix} = 
\begin{pmatrix}
\alpha_{u} & 0 & u_0 \\
0 & \alpha_{v} & v_0 \\
0 & 0 & 1
\end{pmatrix}
\begin{pmatrix}
r_{11} & r_{12} & r_{13} & t_x \\
r_{21} & r_{22} & r_{23} & t_y \\
r_{31} & r_{32} & r_{33} & t_z \\
\end{pmatrix}
\begin{pmatrix}
X_w \\
Y_w \\
Z_w \\
1
\end{pmatrix}
\end{equation}

avec la matrice d'extrinsèque A de la caméra de la forme suivante (équation~\ref{eq:E}) :

\begin{equation}
    A = 
\begin{pmatrix}
r_{11} & r_{12} & r_{13} & t_x \\
r_{21} & r_{22} & r_{23} & t_y \\
r_{31} & r_{32} & r_{33} & t_z \\
\end{pmatrix}
\label{eq:E}
\end{equation}

Et on note H la matrice d'homographie de la forme suivante (équation~\ref{eq:H}) :
\begin{equation}
    H =
\begin{pmatrix}
\alpha_{u} & 0 & u_0 \\
0 & \alpha_{v} & v_0 \\
0 & 0 & 1
\end{pmatrix}
\begin{pmatrix}
r_{11} & r_{12} & r_{13} & t_x \\
r_{21} & r_{22} & r_{23} & t_y \\
r_{31} & r_{32} & r_{33} & t_z \\
\end{pmatrix}
\label{eq:H}
\end{equation}

Pour estimer les paramètres de la matrice d'homographie, nous utiliserons la méthode de Zhang.
La scène qui nous permettra de calibrer la caméra comportera une mire damier (figure~\ref{fig:chessboard}).

\begin{figure}[H]
\centering
\includegraphics[width=7cm]{images/exemple_mire.png}
\caption{Exemple d'une mire damier}
\label{fig:chessboard}
\end{figure}

%-----------------------------------
\subsection{Application de la méthode Zhang pour estimer K et A}
%-----------------------------------

%-----------------------------------
\subsubsection{Calcul de la matrice d'homographie H}
%-----------------------------------

%-----------------------------------
\subsubsection{Calcul de la matrice intrinsèque K}
%-----------------------------------

%-----------------------------------
\subsubsection{Calcul de la matrice extrinsèque A}
%-----------------------------------

%-----------------------------------
\subsubsection{Optimisation par minimisation d'une fonction de perte}
%-----------------------------------

\newpage

%-----------------------------------
\section{Où trouver notre travail ?}
%-----------------------------------

Tout le travail dont il est question dans ce rapport est disponible sur \href{https://github.com/antoinedenovembre/3d_reconstruction}{github}.

\newpage

%-----------------------------------
\newpage
\renewcommand{\refname}{Bibliographie}

\bibliographystyle{plain}
\bibliography{references}

\end{document}
