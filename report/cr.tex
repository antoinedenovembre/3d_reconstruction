% !TeX root = cr.tex

\documentclass[12pt]{article}
\usepackage{svg}
\usepackage[utf8]{inputenc}
\usepackage[T1]{fontenc}
\usepackage[french]{babel}
\usepackage{fancyhdr}
\usepackage{lastpage}
\usepackage{graphicx}
\usepackage{eurosym}
\usepackage{amsmath}
\usepackage{multicol}
\usepackage{geometry}
\usepackage{nccrules}
\usepackage[table]{xcolor}
\usepackage{wrapfig}
\usepackage{pgfgantt}
\usepackage{makecell}
\usepackage{amssymb}
\usepackage{gensymb}
\usepackage{textcomp, gensymb} 
\usepackage{enumitem}
\usepackage{lmodern}
\usepackage{mathrsfs}
\usepackage{textcomp}
\usepackage{multicol}
\usepackage{listings}
\usepackage{media9}
\usepackage{graphicx}
\usepackage{breakurl}
\usepackage{parskip}
\usepackage{float}
\usepackage{listings}
\usepackage{color}
\usepackage{matlab-prettifier}
\PassOptionsToPackage{hyphens}{url}\usepackage{hyperref}
\usepackage{bookmark}

\usetikzlibrary{angles,calc,decorations.pathreplacing}

\title{\textbf{Projet imagerie 3D : Reconstruction d'une scène à partir d'images 2D}}

\author{
    \textsc{Duteyrat Antoine},
    \textsc{Sève Léo}
}

\date{\today}

%----------------------------------------------------------------%
%----------------------------------------------------------------%
%----------------------------------------------------------------%

\definecolor{couleur}{RGB}{0,0,0}
\pagestyle {fancy}

\makeatletter
\let\titre\@title %Variable titre
\let\auteurs\@author %Variable auteurs
\let\date\@date %Variable date
\makeatother


%----------------------------------------------------------------%

%En-tête
\renewcommand{\headrulewidth}{1pt} %Taille du trait
\setlength{\headheight}{45pt}
\fancyhead[L]{\titre}
\fancyhead[R]{}

%Pied de page personnalisé :
\renewcommand{\footrulewidth}{0.5pt} %Taille du trait
\fancyfoot[C]{\thepage\ / \pageref{LastPage}} %PageActuelle / nbrePages au centre

%-----------------------------------------------------------------%
%-----------------------------------------------------------------%
%-----------------------------------------------------------------%

\begin{document}

%-----------------------%
%-----Page de garde-----%
\begin{titlepage}
    \begin{center}
        \vskip 1.5cm
        {\color {couleur}{\huge \bf \titre}}\\[5mm] % Affiche la variable titre
        \vskip 0.5cm
        \begin{figure}[h]
        \centering
        \includegraphics[width=7cm]{images/logo_tse.png}
        \end{figure}
        \vskip 1cm % Saut de ligne
        {\large \auteurs}\\ % Affiche la variable auteurs  
        \vskip 0.5cm % Saut de ligne
        \vfill
        \color{couleur}{\dashrule[1mm]{15cm}{0.5}} % Trait final
        \vskip 0.2cm
        \date % Affiche la variable date
      \end{center}
\end{titlepage}
\clearpage

\tableofcontents

\newpage

%-----------------------------------
\section{Objectif}
%-----------------------------------

L'objectif de ce projet est de reconstruire une scène 3D à partir d'images 2D prises d'une caméra arbitraire.
Pour cela, plusieurs étapes sont nécessaires :

- Calibrer la caméra (calibraton intrinsèque) par la méthode de Zhang (mire plane).

- Prendre plusieurs images d'une scène 3D avec la caméra.

- Triangulation des points 3D à partir des images 2D.

\newpage

%-----------------------------------
\section{Calibration de la caméra}
%-----------------------------------

%-----------------------------------
\subsection{Obtention de la matrice intrinsèque K}
%-----------------------------------

La calibration de la caméra est une étape cruciale pour obtenir des images 3D précises.
Elle permet de déterminer la matrice intrinsèque K, qui est utilisée pour projeter les points 3D du monde réel (repère caméra) sur l'image 2D capturée par la caméra.

Illustrons le calcul de la matrice K par un exemple.

%-----------------------------------
\subsection{Lien matrices K et D, utilisation et projection}
%-----------------------------------

L'objectif de la calibration est de déterminer les valeurs de la matrice intrinsèque K de la caméra de la forme suivante (équation~\ref{eq:K}) :

\begin{equation}
\begin{pmatrix}
\alpha_{u} & \gamma & u_0 \\
0 & \alpha_{v} & v_0 \\
0 & 0 & 1
\end{pmatrix}
\label{eq:K}
\end{equation}

avec :
\begin{equation}
\alpha_{u} = f \times m_{u} \quad \text{et} \quad \alpha_{v} = f \times m_{v}
\label{eq:alpha}
\end{equation}

$\text{où } f \text{ est exprimée en mm et } m_{u}, m_{v} \text{ en pixels/mm.}$

\begin{equation}
m_{u} = \frac{w}{W} \, \text{(en pixels/mm)} \quad \text{et} \quad m_{v} = \frac{h}{H} \, \text{(en pixels/mm)}
\label{eq:m}
\end{equation}

$\text{où } w, h \text{ sont exprimés en pixels et } W, H \text{ en mm.}$

\begin{equation}
u_{0} = \frac{w}{2} \, \text{(en pixels)} \quad \text{et} \quad v_{0} = \frac{h}{2}  \, \text{(en pixels)}
\label{eq:u0v0}
\end{equation}

où :
\begin{itemize}
    \item $\alpha_{u}$ et $\alpha_{v}$ représentent la distance focale \textit{f} en pixels dans les directions verticale et horizontale respectivement (équations~\ref{eq:alpha} et \ref{eq:m}),
    \item $u_0$ et $v_0$ sont les coordonnées du centre optique en pixels (équation~\ref{eq:u0v0}),
    \item $\gamma$ est le coefficient de biais entre l'axe horizontal et l'axe vertical, représentant l'orthogonalité entre les lignes et les colonnes du capteur (0 si orthogonaux),
    \item La dernière ligne est utilisée pour homogénéiser les coordonnées.
\end{itemize}
\vspace{1cm}

Cela implique d'avoir les informations suivantes sur la caméra :
\begin{itemize}
    \item $f$ : distance focale de la caméra,
    \item $w \times h$ : dimensions du capteur de la caméra (en pixels),
    \item $W \times H$ : dimensions du capteur de la caméra (en m) ou les dimensions d'un pixel (en m),
    \item $u_0$ et $v_0$ : coordonnées du centre optique (en pixels), généralement le centre de l'image.
\end{itemize}
\vspace{1cm}

Cette matrice est ensuite utilisée dans la conversion des coordonnées 3D du monde réel en coordonnées 2D de l'image, 
en modèle sténopé sans défaut d'orthogonalité (équation~\ref{eq:projection}).

\begin{equation}
\label{eq:projection}
\begin{pmatrix}
su \\
sv \\
s
\end{pmatrix} = 
\begin{pmatrix}
\alpha_{u} & \gamma & u_0 \\
0 & \alpha_{v} & v_0 \\
0 & 0 & 1
\end{pmatrix}
\begin{pmatrix}
r_{11} & r_{12} & r_{13} & t_x \\
r_{21} & r_{22} & r_{23} & t_y \\
r_{31} & r_{32} & r_{33} & t_z \\
\end{pmatrix}
\begin{pmatrix}
X_w \\
Y_w \\
Z_w \\
1
\end{pmatrix}
\end{equation}

Pour calculer la matrice K, une méthode existe dans la librairie OpenCV, utilisant des mires appelées Charuco (figure~\ref{fig:charuco}).
Ces mires permettent de calibrer les intrinsèques caméra à partir d'une image.
Le module OpenCV effectue la détection de la mire, puis la calibration de la caméra de manière automatique.

\begin{figure}[H]
\centering
\includegraphics[width=7cm]{images/pres_charuco.png}
\caption{Présentation de la mire Charuco utilisée pour la calibration.}
\label{fig:charuco}
\end{figure}

\newpage

%-----------------------------------
\section{Où trouver notre travail ?}
%-----------------------------------

Tout le travail dont il est question dans ce rapport est disponible sur \href{https://github.com/antoinedenovembre/3d_reconstruction}{github}.

\newpage

%-----------------------------------
\section{Bibliographie}
%-----------------------------------

\noindent \textbf{FrenchiDrone} : \textit{Calibration de caméra} [en ligne]. Disponible sur : \url{https://www.frenchidrone.com/calibration-de-camera/} (consulté le 20 mai 2025).

\noindent \textbf{Wikipédia} : \textit{Calibration de caméra} [en ligne]. Disponible sur : \url{https://fr.wikipedia.org/wiki/Calibration_de_cam%C3%A9ra} (consulté le 20 mai 2025).

\end{document}
