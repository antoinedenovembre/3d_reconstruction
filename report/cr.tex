% !TeX root = cr.tex

\documentclass[12pt]{article}
\usepackage{svg}
\usepackage[utf8]{inputenc}
\usepackage[T1]{fontenc}
\usepackage[french]{babel}
\usepackage{fancyhdr}
\usepackage{lastpage}
\usepackage{graphicx}
\usepackage{eurosym}
\usepackage{amsmath}
\usepackage{multicol}
\usepackage{geometry}
\usepackage{nccrules}
\usepackage[table]{xcolor}
\usepackage{wrapfig}
\usepackage{pgfgantt}
\usepackage{makecell}
\usepackage{amssymb}
\usepackage{gensymb}
\usepackage{textcomp, gensymb} 
\usepackage{enumitem}
\usepackage{lmodern}
\usepackage{mathrsfs}
\usepackage{textcomp}
\usepackage{multicol}
\usepackage{listings}
\usepackage{media9}
\usepackage{graphicx}
\usepackage{breakurl}
\usepackage{parskip}
\usepackage{float}
\usepackage{listings}
\usepackage{color}
\usepackage{matlab-prettifier}
\PassOptionsToPackage{hyphens}{url}\usepackage{hyperref}
\usepackage{bookmark}

\usetikzlibrary{angles,calc,decorations.pathreplacing}

\title{\textbf{Projet imagerie 3D : Reconstruction d'une scène à partir d'images 2D}}

\author{
    \textsc{Duteyrat Antoine},
    \textsc{Sève Léo}
}

\date{\today}

%----------------------------------------------------------------%
%----------------------------------------------------------------%
%----------------------------------------------------------------%

\definecolor{couleur}{RGB}{0,0,0}
\pagestyle {fancy}

\makeatletter
\let\titre\@title %Variable titre
\let\auteurs\@author %Variable auteurs
\let\date\@date %Variable date
\makeatother


%----------------------------------------------------------------%

%En-tête
\renewcommand{\headrulewidth}{1pt} %Taille du trait
\setlength{\headheight}{45pt}
\fancyhead[L]{\titre}
\fancyhead[R]{}

%Pied de page personnalisé :
\renewcommand{\footrulewidth}{0.5pt} %Taille du trait
\fancyfoot[C]{\thepage\ / \pageref{LastPage}} %PageActuelle / nbrePages au centre

%-----------------------------------------------------------------%
%-----------------------------------------------------------------%
%-----------------------------------------------------------------%

\begin{document}

%-----------------------%
%-----Page de garde-----%
\begin{titlepage}
    \begin{center}
        \vskip 1.5cm
        {\color {couleur}{\huge \bf \titre}}\\[5mm] % Affiche la variable titre
        \vskip 0.5cm
        \begin{figure}[h]
        \centering
        \includegraphics[width=7cm]{images/logo_tse.png}
        \end{figure}
        \vskip 1cm % Saut de ligne
        {\large \auteurs}\\ % Affiche la variable auteurs  
        \vskip 0.5cm % Saut de ligne
        \vfill
        \color{couleur}{\dashrule[1mm]{15cm}{0.5}} % Trait final
        \vskip 0.2cm
        \date % Affiche la variable date
      \end{center}
\end{titlepage}
\clearpage

\tableofcontents

\newpage

%-----------------------------------
\section{Objectif}
%-----------------------------------

L'objectif de ce projet est de reconstruire une scène 3D à partir d'images 2D prises d'une caméra arbitraire.
Pour cela, plusieurs étapes sont nécessaires :

- Calibrer la caméra (calibraton intrinsèque) par la méthode de Zhang (mire plane).

- Prendre plusieurs images d'une scène 3D avec la caméra.

- Triangulation des points 3D à partir des images 2D.

\newpage

%-----------------------------------
\section{Calibration de la caméra}
%-----------------------------------

L'objectif de la calibration est de déterminer les valeurs de la matrice intrinsèque K de la caméra de la forme suivante :

\begin{equation}
\begin{pmatrix}
su \\
sv \\
s
\end{pmatrix} = 
\begin{pmatrix}
\alpha_{u} & 0 & u_0 \\
0 & \alpha_{v} & v_0 \\
0 & 0 & 1
\end{pmatrix}
\begin{pmatrix}
\alpha_{u} & 0 & u_0 &  \\
0 & \alpha_{v} & v_0 \\
0 & 0 & 1
\end{pmatrix}
\begin{pmatrix}
X_w \\
Y_w \\
Z_w \\
1
\end{pmatrix}
\end{equation}

%-----------------------------------
\section{Où trouver notre travail ?}
%-----------------------------------

Tout le travail dont il est question dans ce rapport est disponible sur \href{https://github.com/antoinedenovembre/colorimetry_project_2}{github}.

\end{document}
